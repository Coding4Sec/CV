%%%%%%%%%%%%%%%%%%%%%%%%%%%%%%%%%%%%%%%%%
% Medium Length Graduate Curriculum Vitae
% LaTeX Template
% Version 1.1 (9/12/12)
%
% This template has been downloaded from:
% http://www.LaTeXTemplates.com
%
% Original author:
% Rensselaer Polytechnic Institute (http://www.rpi.edu/dept/arc/training/latex/resumes/)
%
% Important note:
% This template requires the res.cls file to be in the same directory as the
% .tex file. The res.cls file provides the resume style used for structuring the
% document.
%
%%%%%%%%%%%%%%%%%%%%%%%%%%%%%%%%%%%%%%%%%

%----------------------------------------------------------------------------------------
%	PACKAGES AND OTHER DOCUMENT CONFIGURATIONS
%----------------------------------------------------------------------------------------

\documentclass[margin, 10pt]{res} % Use the res.cls style, the font size can be changed to 11pt or 12pt here

\usepackage{helvet} % Default font is the helvetica postscript font
\usepackage[misc]{ifsym}
\usepackage{wasysym}
\usepackage{fancyhdr}
\usepackage{hyperref}
%\usepackage{newcent} % To change the default font to the new century schoolbook postscript font uncomment this line and comment the one above

\pagestyle{fancy}
\fancyhead{}
\renewcommand{\headrulewidth}{0pt}
\fancyfoot{}
\fancyfoot[RO]{\thepage}
\fancyfoot[LO]{\hspace{-1.35in} Matthew Kenzie}
%\fancyfoot[LO]{\hspace{-1.35in} Matthew Kenzie (176223) - \textit{CV for CERN Fellowship Programme}}

\setlength{\textwidth}{5.3in} % Text width of the document

\newcommand{\Htogg}{\ensuremath{H\rightarrow\gamma\gamma}~}

\begin{document}

%----------------------------------------------------------------------------------------
%	NAME AND ADDRESS SECTION
%----------------------------------------------------------------------------------------

\moveleft.5\hoffset\centerline{\Large\bf\sc Matthew Kenzie} % Your name at the top

\moveleft\hoffset\vbox{\hrule width\resumewidth height 1pt}\smallskip % Horizontal line after name; adjust line thickness by changing the '1pt'

\moveleft.5\hoffset\centerline{\sc Chemin des Croix Rouges 12} % Your address
\moveleft.5\hoffset\centerline{\sc 1007 Lausanne}
%\moveleft.5\hoffset\centerline{\sc SE27 0PZ}
\moveleft.5\hoffset\centerline{\sc Switzerland}
%\moveleft.5\hoffset\centerline{\phone $\;$ \sc +44 7971 264492}
\moveleft.5\hoffset\centerline{\phone $\;$ \sc +48 78 751 91 75}
\moveleft.5\hoffset\centerline{\Letter $\;$  matthew.william.kenzie@cern.ch}
\moveleft.5\hoffset\centerline{\sc \today}

%----------------------------------------------------------------------------------------

\begin{resume}

%----------------------------------------------------------------------------------------
%	RESEARCH INTERESTS SECTION
%----------------------------------------------------------------------------------------

%\section{RESEARCH \\ INTERESTS}

% -- OLD DRAFT
%I have worked in the Higgs to two photon group at CMS for the length of my PhD. The discovery of the new
%boson last year was amazing to be part of. Clearly direct searches for new physics at the LHC are important
%especially given the enhanced cross section of some processes when centre-of-mass energy of the collider increases next year. One of my main interests at the moment is measuring and understanding the couplings and properties of the newly discovered state as this is an excellent way of constraining indirect measurement on new Physics and will cement the Standard Model to higher precision.
% --

% -- FOR CERN fellow application
%I have worked in the Higgs to two photons group at CMS for the length of my PhD.
%It was amazing to be a part of the discovery of the new boson last year.
%My research interests are fairly broad and anything which looks for or probes physics beyond the Standard Model (SM) at the LHC is interesting, whether this be indirectly through rare decays and Higgs coupling measurements or by direct searches for new physics in Supersymmetry (SUSY) or exotics.
%The latter will be of particular interest when the LHC turns back on at a higher centre-of-mass energy given the large increase in production cross-sections of new particles which have a mass in the TeV range. My expertise lies predominantly in using advanced data analysis techniques to extract maximal signal sensitivity, distinguish statistical hypotheses and measure properties. I have considerable experience in analysis software; object reconstruction, event selection and using multivariate techniques and model parametrisations to extract and measure physical quantities.
%The simplest path would be the continuation of my work within the CMS \Htogg group although I am particularly interested in the bigger challenge of applying my skills to direct searches for SUSY which will be especially relevant at the LHC in 2014/2015 when new data at $\sqrt{s}=13$ TeV may be the boundary for natural SUSY.
% --


%----------------------------------------------------------------------------------------
%	EDUCATION AND EMPLOYMENT SECTION
%----------------------------------------------------------------------------------------
\section{EMPLOYMENT}
{\sl Research Fellow} \\
{\bf CERN} \\
May 2014 - Present


\section{EDUCATION}

{\sl PhD High Energy Physics} \\
Departmental thesis prize winner \\
{\bf Imperial College London} \\
October 2010 - April 2014 (Awarded May 2015)

{\sl MSc Theoretical Physics} \\ % - Distinction} \\
Distinction (highest attainable) \\ %\textit{- requires over 80\% in each taught module examination (7 in total) and in the thesis evaluation.} \\
{\bf Imperial College London} \\
October 2009 - September 2010

{\sl BSc (Hons) Physics} \\ % - 1st Class} \\
1st Class (highest attainable) \\ %\textit{- requires an average over 70\%} \\
{\bf Durham University} \\
October 2006 - July 2009

%----------------------------------------------------------------------------------------
%	PROFESSIONAL EXPERIENCE SECTION
%----------------------------------------------------------------------------------------
\section{RESEARCH EXPERIENCE}

{\sl Reseach fellow at CERN, Geneva, working in the LHCb Collaboration}\\
{\bf CERN}
\begin{itemize} \itemsep -2pt
\item Responsible for the LHCb combination of tree level measurements of the CKM angle $\gamma$. This involves writing and maintaining the software framework for the combination, producing results for both the combination and the individual analyses and advising analysts of the input channels on the best approach for $\gamma$ measurements. I am the author for all LHCb $\gamma$ combination papers and public results. These provide the worlds most sensitive measurement of this important parameter.
%\item Responsible for the HFAG combination of measurements of the CKM angle $\gamma$.
\item One of the main authors on a small author list paper on treating model choice uncertainties as discrete nuisance parameters. This is a novel technique and has particular relevance to HEP analyses with large unknown background contributions. It has been used as part of several CMS analyses and also for the LHC Higgs combination. I am working on extensions to this idea with a view to publication in the near future.
\item Provided confidence intervals in the $V_{ub}, \epsilon_{L}$ and $V_{ub}, V_{cb}$ planes for the \textit{Nature Physics} paper on an exclusive measurement of the CKM matrix element $V_{ub}$ using $\Lambda_{b}^{0}$ baryon decays at LHCb.
\item Analyst and main author of the analysis which searched for the rare radiative decays $B_{d(s)}\rightarrow J/\psi \gamma$. I had full responsibility for producing this result.
\item Analyst and main author of a time dependent amplitude analysis of $B_{s}\rightarrow K^{+}\pi^{-}K^{-}\pi^{+}$ decays at LHCb in order to measure
  the CP violating phase in the $B_{s}$ system, $\phi_{s}$. This analysis is ongoing and is expected to finish within the next six months.
\item Worked as a member of the High Level Trigger (HLT) group at LHCb. This includes writing new software tools for real time reconstruction of $D$ and $B$ candidates decaying hadronically in a lifetime unbiased way. This allows for low systematic uncertainties when extracting the lifetime of $B$ and $D$ signal candidates. These trigger lines also provide the necessary input for LHCb's real time alignment and calibration using $D\rightarrow K\pi$ decays.
\item I am the trigger working group liason for $B$ to open charm decays responsible for ensuring the smooth running of the trigger for several imporant LHCb analyses.
\item Worked as an on-call HLT piquet (expert) helping to ensure the smooth running of the trigger system during data taking periods and providing studies of the HLT performance.
\end{itemize}

{\sl PhD student based at CERN, Geneva, working in the CMS Collaboration}\\% \hfill Fall 1990 \\
{\bf Imperial College London}

\indent {\bf Thesis:} {\sl ``Properties of the Higgs-like state around 125~GeV in its decay into two photons at the CMS experiment"} (winner of the Winton Captial prize for best PhD thesis in physics) \\

\begin{itemize} \itemsep -2pt % Reduce space between items
\item Worked as one of the main members of the Higgs to gamma gamma group developing analysis techniques, writing the
code and technical implementation of new ideas and producing the main results for public notes and conferences.
Work has directly contributed towards the discovery of the new particle in 2012.
\item Developed the spin analysis on Higgs to gamma gamma and was editor of the public Physics Analysis Summary which contained the spin analysis results and other property measurements of the new state in its decay to two photons.
\item Have spent much time working on alternatives to the background model which is a significant source of systematic error in low signal-to-background analyses. Developed a method of extracting the background from sidebands which cross checks the main result. Developed a completely new method of estimating the background which can reduce Higgs coupling errors by 10\% in the diphoton channel.
\item Produced the main results and plots for the Higgs to gamma gamma group. Also responsible for the handover of
the final results for statistical combination with the other Higgs channels at CMS.
\item Produced a method, which is the basis of the one currently used, for correcting photon energies, as a function of their position and raw energy, and estimating the per-photon energy resolution.
\item Trained for and worked as Detector on Call for the CMS ECAL team. This consists of week long shifts as the main contact person, available 24/7, for any problem with the CMS ECAL. I have also done shifts in Data Quality Management for the CMS ECAL as a Prompt Feedback Expert and contributed towards the Detector Performance Group at CMS with work on photon and electron clustering.
\end{itemize}

{\sl Masters in Theoretical Physics}\\
{\bf Imperial College London}
\begin{itemize} \itemsep -2pt % Reduce space between items
\item I wrote a thesis on Unparticle Physics which considers the implication of coupling a conformally invariant sector to the Standard Model. The first half concentrated on the technical implication of such a theory whilst the second half discussed the phenomological consequences of it and how experimental signatures could be observed in Hidden Valley like scenarios.
\item Achieved a distinction overall, the highest possible grade, which required over 80\% in the thesis and each examination of the taught courses which included Quantum Field Theory, Group Theory, General Relativity, Differential Geometry, Unification and The Standard Model.
\end{itemize}

{\sl Undergraduate in Industry Program}\\
{\bf Durham University}
\begin{itemize} \itemsep -2pt % Reduce space between items
\item Worked alongside an industrial arm of General Electric developing and testing new strategies for oil and gas pipeline inspection vehicles.
\end{itemize}

{\sl Student Associates Teaching Scheme}\\
{\bf Durham University}
\begin{itemize} \itemsep -2pt % Reduce space between items
\item Assisted teachers in school science lessons. Helpling less able students, improving class participation, teaching parts of lessons, providing demonstrations and promoting higher education.
\end{itemize}

%----------------------------------------------------------------------------------------
% AWARDS
%----------------------------------------------------------------------------------------
\section{AWARDS}

Imperial College, Department of Physics, Thesis Prize 2014 \\
Imperial College Postgraduate Symposium Prize Winner \\

%----------------------------------------------------------------------------------------
%	PUBLICATIONS
%----------------------------------------------------------------------------------------
\section{PUBLICATIONS \& PUBLIC NOTES \newline(with significant contribution)}
The following is a selected list of publications, in chronological order, to which a significant contribution was made. A complete list of 256 publications
can be found at \href{https://inspirehep.net/search?ln=en&ln=en&p=Matthew+Kenzie&of=hb&action_search=Search&sf=earliestdate&so=d&rm=&rg=25&sc=0}{inspirehep.net}

\href{}{LHCb Collaboration, ``Search for the rare decays $B^{0}\rightarrow J/\psi\gamma$ and $B^{0}_{s}\rightarrow J/\psi\gamma$",
\textit{To be submitted to Phys. Rev. D}
\textbf{LHCb-PAPER-2015-044}
}

\href{http://inspirehep.net/record/1373302}{LHCb Collaboration, ``Study of $B^{-}\rightarrow DK^{-}\pi^{+}\pi^{-}$ and $B^{-}\rightarrow D\pi^{-}\pi^{+}\pi^{-}$ decays and determination of the CKM angle $\gamma$",
\textit{Submitted to Phys. Rev. D},
\textbf{LHCb-PAPER-2015-020}
}

\href{http://inspirehep.net/record/1358215}{LHCb Collaboration, ``Determination of the quark coupling strength $|V_{ub}|$ using baryonic decays",
\textit{Nature Phys.}
\textbf{10} (2015) 1038
}

\href{http://inspirehep.net/record/1388239}{LHCb Collaboration, ``Improved constraints on $\gamma$: CKM2014 update",
\textit{LHCb Conference Report}
\textbf{LHCb-CONF-2014-004} (2014)
}

\href{http://inspirehep.net/record/1312971}{P. D. Dauncey, M. Kenzie, N. Wardle, G. J. Davies, ``Handling uncertainties in background shapes: the discrete profiling method",
\textit{JINST}
\textbf{10} (2015) 04, P04015
}

\href{http://inspirehep.net/record/1304454}{CMS Collaboration, ``Observation of the diphoton decay of the Higgs boson and measurement of its properties",
\textit{Eur. Phys. J.}
\textbf{C74} (2014) 10, 3076 - \textbf{192 citations}
}

\href{http://inspirehep.net/record/1224273}{CMS Collaboration, ``Observation of a new boson with mass near 125~GeV in pp collisions at $\sqrt{s}$=7 and 8~TeV",
\textit{JHEP}
\textbf{1306} (2013) 081 - \textbf{345 citations}
}

\href{http://inspirehep.net/record/1124338}{CMS Collaboration, ``Observation of a new boson at a mass of 125 GeV with the CMS experiment at the LHC",
\textit{Phys. Lett. B}
\textbf{716} (2012) 30 - \textbf{4782 citations}
}

\href{http://inspirehep.net/record/1088230}{CMS Collaboration, ``Search for the standard model Higgs boson decaying into two photons in pp collisions at $\sqrt{s}=7$TeV",
\textit{Phys. Lett. B}
\textbf{710} (2012) 403 - \textbf{210 citations}
}

\href{http://inspirehep.net/record/1260840}{CMS Collaboration, ``Properties of the observed Higgs-like resonance decaying into two photons",
\textit{CMS Physics Analysis Summary}
\textbf{CMS-PAS-HIG-13-001} (2013)% - \textbf{49 citations}
}

%----------------------------------------------------------------------------------------
%	CONFERENCES AND SCHOOLS
%----------------------------------------------------------------------------------------

\section{CONFERENCES, TALKS, SEMINARS \& SCHOOLS}

HEP Seminar, Liverpool University 2015 \\
\textbf{Treating model choice uncertainties as discrete nuisances parameters} \\
\textit{Invited seminar talk}

School on Precision Measurements, Technische Universit\"{a}t Dortmund 2015 \\
\textbf{Systematic Uncertainties} \\
\textit{Invited summer school lecture}


LHCP, St. Petersburg 2015 \\
\textbf{B properties and CP violation} (On behalf of LHCb collaboration) \\
\textit{Conference plenary invited talk}

B2DX Fitter Workshop, Padova 2015 \\
\textbf{LHCb $\mathbf{\gamma}$ combination}\\
\textit{Invited workshop talk}

53rd International Winter Meeting on Nuclear Physics, Bormio 2015 \\
\textbf{LHCb Overview} (On behalf of LHCb collaboration) \\
\textit{Conference plenary invited talk}

ICHEP, Valencia 2014 \\
\textbf{Higgs to two photons at CMS} (On behalf of the CMS collaboration) \\
\textit{Conference parallel invited talk}

%CMS Approval, CERN 2014 \\
%\textbf{Higgs to two photons legacy result} \\
%\textit{Approval talk}

Imperial College Postgraduate Symposium, London 2013\\
\textbf{Higgs to two photons at CMS} \\
\textit{Invited symposium talk} (Prize winner)

Higgs Hunting, Orsay 2013 \\
\textbf{Higgs to two photons at CMS} (On behalf of CMS collaboration) \\
\textit{Conference plenary invited talk}

LHC Collider Cross Talk, CERN 2013 \\
\textbf{Higgs to two photons at CMS} (On behalf of CMS collaboration) \\
\textit{Invited workshop chalk and talk}

%CMS UK, Oxford 2013 \\
%\textbf{Jackknifing the Higgs to gamma gamma analysis} \\
%\textit{Invited workshop talk}

%CERN Summer School, Arequipa 2013 \\
%\textbf{Higgs to two photons at CMS} \\
%\textit{Poster}

%RAL High Energy Physics Summer School, Oxford 2011 \\
%\textbf{Higgs to two photons at CMS} \\
%\textit{Poster}

%----------------------------------------------------------------------------------------
%	OUTREACH SECTION
%----------------------------------------------------------------------------------------
\section{OUTREACH}

TEDx Institue Le Rosey, Rolle 2015 \\
\textbf{The future} \\
\textit{Invited speaker}

Brillantmont International School, Lausanne 2015 \\
\textbf{Why do we do science?} and \textbf{Finding the Higgs boson} \\
\textit{Invited speaker and Science Fair Judge}
%Brillantmont International School, Lausanne 2015, \textbf{Science Fair Judge}

International School of Lausanne, Lausanne 2015 \\
\textbf{History of time}, \textbf{Finding the Higgs boson} and \textbf{The theory of knowledge} \\
\textit{Invited speaker and Science Fair Judge}
%International School of Lausanne, Lausanne 2015, \textbf{Science Fair Judge}

Perse School, Cambridge 2013 \\
\textbf{How we found the Higgs} \\
\textit{Invited speaker}

Royal Society, London 2013 \\
\textbf{Discovering the Higgs Boson} \\
\textit{Royal Society Fair demonstrator}

%----------------------------------------------------------------------------------------
%	COMPUTER SKILLS SECTION
%----------------------------------------------------------------------------------------
%\section{COMPUTER \\ SKILLS}

%{\sl Languages:}
%C, C++, python, bash, HTML \\
%{\sl Software:}
%\LaTeX, Microsoft Word, Excel, Powerpoint, ROOT, RooFit, RooStats \\
%{\sl Operating Systems:}
%Windows, Mac OS, Linux, SLC

%----------------------------------------------------------------------------------------
%	REFERENCES
%----------------------------------------------------------------------------------------
\newpage
\section{REFERENCES}
\textbf{Prof. Paul Dauncey} (PhD Supervisor) \\
\textit{Head of Imperial College High Energy Physics Group} \\
Imperial College London, Blackett Laboratory, Prince Consort Rd, London, SW7 2BW, UK \\
\texttt{p.dauncey@imperial.ac.uk}

\textbf{Prof. Guy Wilkinson} \\
\textit{LHCb Collaboration Spokesperson} \\
CERN, Geneva 23, CH-1211, Switzerland \\
\texttt{guy.wilkinson@cern.ch}

\textbf{Dr. Patrick Koppenburg} \\
\textit{LHCb Collaboration Physics Coordinator} \\
CERN, Geneva 23, CH-1211, Switzerland \\
\texttt{patrick.koppenburg@cern.ch}

%\textbf{Prof. Jim Virdee} \\
%\textit{Professor of High Energy Physics} \\
%Imperial College London, Blackett Laboratory, Prince Consort Rd, London, SW7 2BW, UK \\
%\texttt{t.virdee@imperial.ac.uk}

%\textbf{Dr. Andr\'{e} David Tinoco Mendes} \\
%\textit{Research physicist} \\
%CERN, Geneva 23, CH-1211, Switzerland \\
%\texttt{andre.david@cern.ch}

%\textbf{Dr. Chris Seez} \\
%\textit{Research physicist} \\
%Imperial College London, Blackett Laboratory, Prince Consort Rd, London, SW7 2BW, UK \\
%\texttt{chris.seez@cern.ch}

%\textbf{Dr. Paolo Meridiani} \\
%\textit{Research physicist} \\
%Istituto Nazionale di Fisica Nucleare, Piazza dei Caprettari, 70, 00186 Roma, Italy \\
%\texttt{paolo.meridiani@cern.ch}

%\textbf{Dr. Pasquale Musella} \\
%\textit{Research physicist} \\
%CERN, Geneva 23, CH-1211, Switzerland \\
%\texttt{pasquale.musella@cern.ch}

\end{resume}
\end{document}
